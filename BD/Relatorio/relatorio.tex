\documentclass[pdftex,12pt,a4paper]{report}

\usepackage[portuguese,english]{babel}
\usepackage[T1]{fontenc} 
\usepackage[utf8]{inputenc}
\usepackage[pdftex]{graphicx}
\usepackage{minitoc}
\usepackage{hyperref}
\usepackage{indentfirst}
\usepackage[compact]{titlesec}
\usepackage{fancyhdr}
\usepackage{caption}
\usepackage{pgfplots}
\usepackage{pgfplotstable}
\usepackage{fixltx2e}
\usepackage{mathtools}
\usepackage{fancyhdr}
\usepackage{listings}
\usepackage{color}
\definecolor{CadetBlue}{rgb}{0.37, 0.62, 0.63}
\definecolor{OliveGreen}{rgb}{0,0.6,0}

\lstset{
  breaklines=true,                                     % line wrapping on
  language=SQL,
  frame=ltrb,
  framesep=5pt,
  basicstyle=\normalsize,
  keywordstyle=\ttfamily\color{OliveGreen},
  identifierstyle=\ttfamily\color{CadetBlue}\bfseries,
  commentstyle=\color{Brown},
  stringstyle=\ttfamily,
  showstringspaces=ture
}

\pagestyle{fancy}
\renewcommand*\thesection{\thechapter\arabic{section}}
\newcommand{\HRule}{\rule{\linewidth}{0.5mm}}
\begin{document}

\begin{titlepage}

\begin{center}

\includegraphics[width=0.15\textwidth]{./logo}\\[0.5cm]    

\textsc{\large Universidade de Aveiro \\[1cm]\large departamento de electrónica, telecomunicações e informática}\\[1cm]

\textsc{\large{42532}\large - Base de Dados \\[1cm]}

\HRule \\[0.5cm]
{ \huge \bfseries Football Club}\\[0.4cm]
{ \large \bfseries Trabalho Prático Final}\\[0.4cm]
\HRule \\[1cm]

\textsc{\small{8240 - MESTRADO INTEGRADO EM ENGENHARIA DE COMPUTADORES E TELEMÁTICA}}\\[1cm]

\begin{minipage}{0.4\textwidth}

\begin{flushleft} \large
\href{mailto:rafael.ferreira@ua.pt}{António Rafael da \\ Costa Ferreira }
 \small{\\NMec: 67405 | P4G5}
\end{flushleft}
\end{minipage}
\begin{minipage}{0.4\textwidth}

\begin{flushright} \large
\href{mailto:rodrigocunha@ua.pt}{Rodrigo Lopes \\ da Cunha}
\small{\\NMec: 67800 | P4G5}
\end{flushright}
\end{minipage}\\[1cm]

{\large Docente: Carlos Manuel Azevedo Costa   }\\[0.5cm]

\vfill

{\large Junho de 2015 \\ 2014-2015}

\end{center}

\end{titlepage} %Titulo do Relatorio
\renewcommand{\headrulewidth}{0pt}

%Cabeçalhos de rodapé
\fancyhead{}
\fancyfoot{}
\lhead{Football Club - Trabalho Prático Final}
\rhead{BD - 2014/2015}
\lfoot{\textit{P4G5} \\ Rafael Ferreira nmec: 67405 \\ Rodrigo Cunha nmec: 67800}
\rfoot{\thepage}

%Renomear Comandos
\renewcommand*\contentsname{Conteúdos}
\renewcommand*\figurename{Figura}
\renewcommand*\tablename{Tabela}

%Conteúdos, dar paragrafo
\tableofcontents
%Headers
\renewcommand{\headrulewidth}{0.15pt}
\renewcommand{\thechapter}{}

\clearpage

\section{Introdução}
% o que, porquê e o objetivo
O trabalho proposto para o projeto da unidade curricular de Base de Dados é uma plataforma de gestão de um clube de futebol. Usando os conhecimentos adquiridos, propôs-se o desenvolvimento deste projeto visto que o futebol é uma modalidade mundial, envolvendo vários tipos de interesse. 

O objetivo desta base de dados desenvolvida é permitir a gestão de todos os processos de um clube de futebol, como será visto mais à frente.

Além de se ter desenvolvido esta base de dados, desenvolveu-se uma aplicação WPF C\# para permitir a manipulação dos dados da base de dados de forma mais simplificada para um utilizador final.

Esta base de dados deve fornecer ferramentas que permitam a criação, remoção, alteração e consulta da base de dados de forma segura, eficiente e robusta.

O relatório reflete todos os passos e decisões tomadas na criação da base de dados que sustenta o projeto bem como uma descrição das capacidades da aplicação desenvolvida para o cliente.

Para a criação deste projeto foi seguido o processo leccionado nas aulas, sendo estas as seguintes fases do processo: análise de requisitos, desenho conceptual, desenho do esquema lógico, desenho do esquema físico e administração.

\newpage
\section{Análise de Requisitos}

A análise de requisitos foi uma das partes mais importantes do processo de concepção do projeto uma vez que ajudou a ter uma visão clara do que o sistema teria de suportar. Após realizado um "brainstorming", estas são as características que o sistema deve suportar:

Uma \textbf{pessoa} é identificada por um nome, B.I. (sendo que este B.I. é único), endereço, NIF, Sexo, Data de Nascimento e Nacionalidade. Esta pessoa pode ser uma Pessoa que pertença ao pessoal interno do clube (Staff) ou ser sócio do clube. 

Uma \textbf{pessoa interna} ao clube tem salário e um ID que é automaticamente atribuído e o identifica dentro do clube.

Um \textbf{sócio do clube} tem um nº de sócio, o ano até que as suas cotas estão pagas (são cotas anuais) e um valor de cotas que tem de pagar todos os anos. Um sócio pode ter ou não um \textbf{lugar anual}, tendo este, um valor, data de início, duração, Nº Lugar e Nº Fila e ID da secção.

Um \textbf{lugar} tem um nº de lugar e fila. Uma \textbf{secção} tem um ID de secção e tipo. 

Um \textbf{jogador} é uma pessoa interna ao clube e é identificado com um ID da federação, peso e altura. Este joga em equipas do clube.

Um \textbf{treinador} é uma pessoa interna do clube e é identificado com um ID da federação e cargo. Este tem equipas do clube.

Uma \textbf{equipa} tem uma idade máxima de jogadores que podem pertencer à mesma e um nome único.

Uma equipa pode ter \textbf{treinos} que são caracterizados por uma data e uma hora e são realizados num determinado campo.

Um \textbf{campo} tem um endereço e um ID.

Uma pessoa interna ao clube (\textbf{Staff}) tem um cargo e pode trabalhar um Departamento.

Um \textbf{departamento} tem um endereço, ID de departamento e um nome.

\newpage

Foram também registadas algumas especificações para o desenho conceptual:

- Uma pessoa pode ser um sócio e uma pessoa interna.

- Uma pessoa interna ao clube pode ser um jogador, um coach ou um membro do staff.

- Um sócio pode ter vários lugares anuais mas um lugar anual apenas pertence a um membro.

- Um lugar pode ter vários lugares anuais mas um lugar anual apenas tem um lugar.

- Uma secção pode ter vários lugares mas um lugar pode ter apenas uma secção.

- Um membro do staff apenas pode trabalhar num departamento e um departamento pode ter vários membros do staff.

- Um jogador pode jogar em várias equipas e uma equipa pode ter vários jogadores.

- Um treinador pode jogar em várias equipas e uma equipa pode ter vários treinadores.

- Uma equipa pode ter vários treinos mas um treino apenas pode ter uma equipa.

- Um treino apenas pode ter um campo e um campo pode ter vários treinos.

\newpage
\section{Diagrama entidade relação}
Depois da análise de requisitos desenhou-se o diagrama entidade relação do nosso sistema. Este desenho foi descrito através de um diagrama ER\ref{fig:eer}. No diagrama, foram definidas entidades, atributos, relações, cardinalidades e dependências.

\begin{figure}[!htb]
 \includegraphics[width=135mm,scale=1]{EER.pdf}
 \caption{\\Diagrama entidade relação}\label{fig:eer}
\end{figure}

As entidades e os seus atributos correspondem à análise de requisitos realizada anteriormente. As relações são todas binárias.

\newpage
\section{Esquema Relacional da BD}
Após a construção do nosso desenho conceptual procedeu-se à elaboração do Modelo Relacional. Este modelo foi construído tendo por base o diagrama entidade relação e as regras para a realização desta tarefa. Cada entidade e cada relação irá gerar uma única tabela e após realizados os passos para conversão do desenho conceptual no modelo relacional 
foi criado o modelo relacional\ref{fig:mr}.

\begin{figure}[!htb]
 \includegraphics[width=100mm,scale=1]{modelo_relacional.pdf}
 \caption{\\Modelo relacional}\label{fig:mr}
\end{figure}

\newpage
\section{Normalização}

\newpage
\section{Índices}
Com o aumento do volume de dados, os pedidos de consulta começam a ter tempos de resposta maiores, pois os dados que existem nas tabelas encontram-se desorganizados. Para contrariar este aumento dos tempos de resposta devemos manter as tabelas organizadas de forma a que as consultas sejam efetuadas mais rapidamente. A utilização de índices nos campos que mais frequentemente serão utilizados é a maneira indicada, onde um ponteiro é criado para a posição real de cada registo.
\\

	Para tornar as nossas pesquisas mais eficientes introduzimos vários índices que achamos necessários (pesquisas feitas por dados não primários), do tipo NonClustered, pois os índices do tipo Clustered são automaticamente criados aquando da criação de uma tabela pois vai corresponder às chaves primárias.
\\

Índice "indexactiveseat", exemplo:
\\

\begin{lstlisting}
go

CREATE NONCLUSTERED INDEX indexactiveseat ON football.seat(active)
WITH (FILLFACTOR=75,pad_index=ON);
\end{lstlisting}
 \vspace{0,5in}
Restantes índices (em anexo):
\\

indexsalaryinternalpeople

indexheightplayer

indexweigthplayer

\end{document}