\documentclass[pdftex,12pt,a4paper]{report}

\usepackage[portuguese,english]{babel}
\usepackage[T1]{fontenc} 
\usepackage[utf8]{inputenc}
\usepackage[pdftex]{graphicx}
\usepackage{minitoc}
\usepackage{hyperref}
\usepackage{indentfirst}
\usepackage[compact]{titlesec}
\usepackage{fancyhdr}
\usepackage{caption}
\usepackage{pgfplots}
\usepackage{pgfplotstable}
\usepackage{fixltx2e}
\usepackage{mathtools}
\pagestyle{fancy}
\fancyhead{}
\renewcommand*\thesection{\thechapter\arabic{section}}
\newcommand{\HRule}{\rule{\linewidth}{0.5mm}}
\begin{document}

\begin{titlepage}

\begin{center}

\includegraphics[width=0.15\textwidth]{./logo}\\[0.5cm]    

\textsc{\large Universidade de Aveiro \\[1cm]\large departamento de electrónica, telecomunicações e informática}\\[1cm]

\textsc{\large{42532}\large - Base de Dados \\[1cm]}

\HRule \\[0.5cm]
{ \huge \bfseries Football Club}\\[0.4cm]
{ \large \bfseries Trabalho Prático Final}\\[0.4cm]
\HRule \\[1cm]

\textsc{\small{8240 - MESTRADO INTEGRADO EM ENGENHARIA DE COMPUTADORES E TELEMÁTICA}}\\[1cm]

\begin{minipage}{0.4\textwidth}

\begin{flushleft} \large
\href{mailto:rafael.ferreira@ua.pt}{António Rafael da \\ Costa Ferreira }
 \small{\\NMec: 67405 | P4G5}
\end{flushleft}
\end{minipage}
\begin{minipage}{0.4\textwidth}

\begin{flushright} \large
\href{mailto:rodrigocunha@ua.pt}{Rodrigo Lopes \\ da Cunha}
\small{\\NMec: 67800 | P4G5}
\end{flushright}
\end{minipage}\\[1cm]

{\large Docente: Carlos Manuel Azevedo Costa   }\\[0.5cm]

\vfill

{\large Junho de 2015 \\ 2014-2015}

\end{center}

\end{titlepage} %Titulo do Relatorio
\renewcommand{\headrulewidth}{0pt}

%Parte do resumo!
\vspace*{\fill}
\textbf{Introdução:}
\begingroup
Neste trabalho 2 pretende-se estudar as características e utilizações das portas lógicas NMOS, PMOS, CMOS e do BJT e de um díodo.
\endgroup
\vspace*{\fill}
%Parte do resumo!
\newpage

%Renomear Comandos
\renewcommand*\contentsname{Conteúdos}
\renewcommand*\figurename{Figura}
\renewcommand*\tablename{Tabela}

%Conteúdos, dar paragrafo
\tableofcontents
%Headers
\renewcommand{\headrulewidth}{0.15pt}
\lhead{Relatório Trabalho Prático Nº2}
\renewcommand{\thechapter}{}

\clearpage

\section{Parte I - Análise de Circuitos com Díodos}

\end{document}